\begin{figure}[t]
    \centering
    \begin{subfigure}[c]{0.48\textwidth}
        \centering
        \includegraphics[width=\linewidth]{media/wiki_viz/ollm_example.png}
        \caption{OLLM}
        \label{fig:wiki-examples:ollm}
    \end{subfigure}%
    \hfill%
    \begin{subfigure}[c]{0.48\textwidth}
        \centering
        \includegraphics[width=\linewidth]{media/wiki_viz/prompting_3_example.png}
        \caption{Three-shot}
        \label{fig:wiki-examples:prompting}
    \end{subfigure}%
    \hfill%
    \begin{subfigure}[c]{0.48\textwidth}
        \centering
        \includegraphics[width=\linewidth]{media/wiki_viz/hearst_example.png}
        \caption{Hearst}
    \end{subfigure}%
    \hfill%
    \begin{subfigure}[c]{0.48\textwidth}
        \centering
        \includegraphics[width=\linewidth]{media/wiki_viz/rebel_example.png}
        \caption{REBEL}
    \end{subfigure}%
    \caption{Examples snapshots of subgraphs of the generated ontologies for Wikipedia. Edges coloured in \textcolor{blue}{\textbf{blue}} are present in the test split, while those in \textcolor{red}{\textbf{red}} are not. Qualitatively, the relations in \name is the most precise in terms of its semantics. The full pictures can be found in \cref{appendix:viz-wiki}.}
    \label{fig:wiki-examples}
\end{figure}