\begin{figure}[t]
    \centering
    \begin{tikzpicture}
        [
            >=latex,
            doc/.style={
                    rectangle,
                    rounded corners=2pt,
                    minimum width=10pt,
                    minimum height=14pt,
                    outer sep=2pt,
                },
            llm/.style={
                    rectangle,
                    rounded corners=2pt,
                    fill={black!20},
                    minimum width=40pt,
                    minimum height=24pt,
                    outer sep=3pt,
                },
            concept/.style={
                    circle,
                    fill={black!30},
                    inner sep=1.25pt,
                    outer sep=2.5pt,
                },
            every node/.append style={font=\sffamily},
            pics/.cd,
            % Marque croix en diagonale
            Cross/.style args={#1 and #2}{%
                    code = {%
                            \draw[#2,rotate=45,scale=1.4,very thick]
                            (0,#1 pt) -- (0,-#1 pt) ;
                            \draw[#2,rotate=-45,scale=1.4,very thick]
                            (0,#1 pt) -- (0,-#1 pt) ;
                        }
                },
            Cross/.default={2.5 and gray!25!black},
        ]
        \newcommand{\nodeDist}{0.8}
        \newcommand{\nodeDistShort}{0.6}
        \newcommand{\angleA}{15}
        \newcommand{\angleB}{58}
        \definecolor{c0}{RGB}{150, 150, 150}
        \definecolor{c1}{RGB}{255, 93, 82}
        \definecolor{c2}{RGB}{94, 118, 255}
        \definecolor{c3}{RGB}{118, 177, 207}
        \definecolor{c4}{RGB}{215, 140, 255}
        \definecolor{c5}{RGB}{255, 195, 84}
        \definecolor{c6}{RGB}{64, 194, 62}

        % PART A: Ontology
        \node[concept,color=c0] (root) {};
        \node[concept,color=c4] (l1) at ($(root) + ({-90+\angleA}:{\nodeDist+0.55})$) {};
        \node[concept,color=c6] (l2) at ($(root) + ({90-\angleA}:{\nodeDist+0.55})$) {};
        \node[concept,color=c1] (l11) at ($(l1) + ({90-\angleB}:{\nodeDist})$) {};
        \node[concept,color=c2] (l12) at ($(l1) + ({-90+\angleB}:{\nodeDist})$) {};
        \node[concept,color=c3] (l21) at ($(l2) + ({90-\angleB}:{\nodeDist})$) {};
        \node[concept,color=c5] (l22) at ($(l2) + ({-90+\angleB}:{\nodeDist})$) {};
        % \node[concept] (l1) at ($(root) + ({180+\angleA}:{\nodeDist})$) {};
        % \node[concept] (l2) at ($(root) + ({-\angleA}:{\nodeDist})$) {};
        % \node[concept] (l11) at ($(l1) + ({180+\angleB}:{\nodeDist})$) {};
        % \node[concept] (l12) at ($(l1) + ({-\angleB}:{\nodeDist})$) {};
        % \node[concept] (l21) at ($(l2) + ({180+\angleB}:{\nodeDist})$) {};
        % \node[concept] (l22) at ($(l2) + ({-\angleB}:{\nodeDist})$) {};
        \draw[->] (root) -- (l1);
        \draw[->] (root) -- (l2);
        \draw[->] (l1) -- (l11);
        \draw[->] (l1) -- (l12);
        \draw[->] (l2) -- (l21);
        \draw[->] (l2) -- (l22);

        \node[anchor=west,inner sep=0,align=left] (caption) at ($(root.west) + (0.4, 2.55)$) {\small Dataset};

        \newcommand{\docSpace}{0.66}
        \node[doc,fill=c1] (doc1) at ($(l11) + ({90-\angleB-10}:{\nodeDist})$) {};
        \node[doc,fill=c2] (doc2) at ($(l12) + ({-90+\angleB+10}:{\nodeDist})$) {};
        \node[doc,fill=c3] (doc3) at ($(l21) + ({90-\angleB-10}:{\nodeDist})$) {};
        % \node[doc,fill=c4] (doc4) at ($(l11) + (\docSpace, -1.5)$) {};
        \node[doc,fill=c5] (doc7) at ($(l22) + ({-90+\angleB+10}:{\nodeDist})$) {};
        \node[doc,rotate=90] (doc5) at ($(doc1)!0.5!(doc2)$) {...};
        \node[doc,rotate=90] (doc6) at ($(doc3)!0.5!(doc7)$) {...};
        \draw[dashdotted] (l11) to (doc1);
        \draw[dashdotted] (l12) to (doc2);
        \draw[dashdotted] (l21) to (doc3);
        \draw[dashdotted] (l22) to (doc7);
        % \draw[dashdotted,bend right] (l1) to (doc4);

        % line to separate the ontology into train and test
        % \draw[dashed] ($(root) + (0, 0.4)$) -- ($(root) + (0, -2.6)$);
        % \node at ($(root) + (1.1, -0.1)$) {Train};
        % \node at ($(root) + (-1.1, -0.1)$) {Test};

        % \draw[dashed] ($(root) + (-0.4, 0)$) -- ($(root) + (2.0, 0)$);
        % \node[anchor=west,inner sep=0] at ($(root.west) + (0, 1.1)$) {\small Train};
        % \node[anchor=west,inner sep=0] at ($(root.west) + (0, -1.5)$) {\small Test};

        % PART B: Training
        \newcommand{\vertSpace}{0.2}
        \newcommand{\trainSpace}{0.7}
        % \node[doc,fill=c3] (input) at ($(doc3) + (1.0, 0)$) {};
        % \node[doc,fill=c5] (inputp) at ($(doc7) + (1.0, 0)$) {};

        \node[llm,anchor=west] (llm) at ($(doc3)!0.5!(doc7) + (1, 0)$) {\small LLM};
        \node[anchor=west,align=center,rounded corners=1.5pt,draw,outer sep=3pt] (loss) at ($(llm.east) + (\trainSpace + 0.2, 0)$) {\small Mask-regularised loss};

        \begin{scope}[local bounding box=target1,shift={($(loss.east) + (0.7, 0.2)$)},outer sep=3pt]
            \node[concept,color=c0] (sgRoot) {};
            \node[concept,color=c6] (sg2) at ($(sgRoot) + ({90-\angleA-20}:{\nodeDistShort})$) {};
            \node[concept,color=c3] (sg21) at ($(sg2) + ({90-\angleB}:{\nodeDistShort})$) {};
            \draw[->] (sgRoot) -- (sg2);
            \draw[->] (sg2) -- (sg21);
        \end{scope}

        \begin{scope}[local bounding box=target2,shift={($(loss.east) + (0.7, -0.7)$)},outer sep=3pt]
            \node[concept,color=c0] (sgRoot) {};
            \node[concept,color=c6] (sg2) at ($(sgRoot) + ({90-\angleA-20}:{\nodeDistShort})$) {};
            \node[concept,color=c5] (sg22) at ($(sg2) + ({-90+\angleB}:{\nodeDistShort})$) {};
            \draw[->] (sgRoot) -- (sg2);
            \draw[->] (sg2) -- (sg22);
        \end{scope}

        \coordinate (out) at ($(llm.west) + (0, \vertSpace)$);
        \draw[color=c3] (doc3) -| ($(doc3.east)!0.5!(out)$) |- (out);
        \draw[->,color=c3] ($(llm.east) + (0, \vertSpace)$) -- ($(loss.west) + (0, \vertSpace)$);

        \coordinate (out) at ($(llm.west) + (0, -\vertSpace)$);
        \draw[color=c5] (doc7) -| ($(doc7.east)!0.5!(out)$) |- (out);
        \draw[->,color=c5] ($(llm.east) + (0, -\vertSpace)$) -- ($(loss.west) + (0, -\vertSpace)$);

        \draw[->,color=c3,bend right] (target1) to (loss);
        \draw[->,color=c5,bend left] (target2) to (loss);
        \draw[->,bend right,color=c3] (loss) to node [midway,above,align=center,color=black] {\scriptsize Backpropagate} (llm);
        \draw[->,bend left,color=c5] (loss) to (llm);
        \node at ($(target1) + (0, 0.6)$) {\scriptsize Target};

        % PART C: Inference
        \node[llm,anchor=west] (llm1) at ($(doc1)!0.5!(doc2) + (1.0, 0)$) {\small LLM};
        % \node[doc,fill=c1,anchor=east] (input1) at ($(llm1.west) + (-\trainSpace, 0.8)$) {};

        \begin{scope}[local bounding box=sg1,shift={($(llm1.east) + (\trainSpace + 0.2, 0.6)$)}]
            \node[concept,color=c0] (sgRoot1) {};
            \node[concept,color=c4] (sg11) at ($(sgRoot1) + ({-90+\angleA+20}:{\nodeDistShort})$) {};
            \node[concept,color=c6] (sg12) at ($(sgRoot1) + ({90-\angleA-20}:{\nodeDistShort})$) {};
            \node[concept,color=c1] (sg112) at ($(sg11) + ({90-\angleB}:{\nodeDistShort})$) {};
            \draw[->] (sgRoot1) -- (sg12);
            \draw[->] (sgRoot1) -- (sg11);
            \draw[->] (sg11) -- (sg112);
        \end{scope}

        \begin{scope}[local bounding box=sg2,shift={($(llm1.east) + (\trainSpace + 0.2, -0.5)$)}]
            \node[concept,color=c0] (sgRoot2) {};
            \node[concept,color=c4] (sg21) at ($(sgRoot2) + ({-90+\angleA+20}:{\nodeDistShort})$) {};
            \node[concept,color=c2] (sg211) at ($(sg21) + ({-90+\angleB}:{\nodeDistShort})$) {};
            \draw[->] (sgRoot2) -- (sg21);
            \draw[->] (sg21) -- (sg211);
        \end{scope}


        % \node (dots) at ($(input1)!0.5!(input2)$) {...};
        % \node (dots2) at ($(dots) + (4.0, 0)$) {...};
        \node[rotate=90] at ($(sg1)!0.5!(sg2) + (0, -0.15)$) {...};
        \draw[color=c1] (doc1) -| ($(doc1.east)!0.5!(llm1.west)$) |- ($(llm1.west) + (0, \vertSpace)$);
        \coordinate (out1) at ($(llm1.east) + (\trainSpace, 0.6)$);
        \draw[->,color=c1] ($(llm1.east) + (0, \vertSpace)$) -| ($(llm1.east)!0.5!(out1)$) |- (out1);
        \draw[color=c2] (doc2) -| ($(doc2.east)!0.5!(llm1.west)$) |- ($(llm1.west) + (0, -\vertSpace)$);
        \coordinate (out2) at ($(llm1.east) + (\trainSpace, -0.6)$);
        \draw[->,color=c2] ($(llm1.east) + (0, -\vertSpace)$) -| ($(llm1.east)!0.5!(out2)$) |- (out2);


        % \node[doc,fill=c1] (input1) at ($(input) + (0, -2.0)$) {};
        % \node[llm] (llm1) at ($(input1) + (\trainSpace, 0)$) {LLM};
        % \node[concept] (sgRoot1) at ($(llm1) + (\trainSpace + 0.8, 0.6)$) {};
        % \node[concept] (sg11) at ($(sgRoot1) + ({180+\angleA}:{\nodeDist})$) {};
        % \node[concept] (sg12) at ($(sgRoot1) + ({-\angleA}:{\nodeDist})$) {};
        % \node[concept] (sg111) at ($(sg11) + ({180+\angleB}:{\nodeDist})$) {};
        % \draw[->] (input1) -- (llm1);
        % \draw[->] (llm1) -- ($(llm1) + (\trainSpace-0.25, 0)$);

        % \node[doc,fill=c2] (input2) at ($(input1) + (0, -1.7)$) {};
        % \node[llm] (llm2) at ($(input2) + (\trainSpace, 0)$) {LLM};
        % \node[concept] (sgRoot2) at ($(llm2) + (\trainSpace + 0.8, 0.6)$) {};
        % \node[concept] (sg21) at ($(sgRoot2) + ({180+\angleA}:{\nodeDist})$) {};
        % \node[concept] (sg212) at ($(sg21) + ({-\angleB}:{\nodeDist})$) {};
        % \draw[->] (input2) -- (llm2);
        % \draw[->] (llm2) -- ($(llm2) + (\trainSpace-0.25, 0)$);

        \node[concept,color=c0] (testOutRoot) at ($(doc1)!0.5!(doc2) + (7.1, 0.0)$) {};
        \node[concept,color=c4] (testOut1) at ($(testOutRoot) + ({-90+\angleA+20}:{\nodeDist})$) {};
        \node[concept,color=c2] (testOut11) at ($(testOut1) + ({-90+\angleB}:{\nodeDist})$) {};
        \node[concept,color=c1] (testOut12) at ($(testOut1) + ({90-\angleB}:{\nodeDist})$) {};
        \node[concept,color=c6] (testOut2) at ($(testOutRoot) + ({90-\angleA-20}:{\nodeDist})$) {};
        \node at ($(testOutRoot) + (0.4, 0.9)$) {\scriptsize Output};
        \draw[->] (testOutRoot) -- (testOut1);
        \draw[->] (testOut1) -- (testOut11);
        \draw[->] (testOut1) -- (testOut12);
        \draw[->] (testOutRoot) -- pic[midway,-,rotate=60] {Cross={3.5 and red}} (testOut2);
        \draw[->] ($(doc1)!0.5!(doc2) + (4.6, 0)$) -- node [midway,above,align=center,draw,rounded corners=0.8pt,inner sep=1.4pt,outer sep=3.2pt] {\scriptsize Sum and prune} ($(doc1)!0.5!(doc2) + (6.8, 0)$);

        % PART D: Evaluation
        \coordinate (start) at ($(l1) + (0, -0.4)$);
        \coordinate (inter) at ($(start) + (4.5, -1.1)$);
        \coordinate (end) at ($(testOut1) + (0, -0.3)$);
        \draw[<->,rounded corners] (start) |- (inter) -| (end);
        \node[below] at (inter) {\footnotesize Gold standard evaluation};

        % Draw a separation line between Part B and Part C
        \draw[color={black!30},dashed] ($(root) + (-0.85, 0)$) -- ($(root) + (12.65, 0)$);
        % Caption Part B
        \node[color={black!60},inner sep=0,rotate=90] at ($(l2) + (-0.8, 0)$) {\small Training};
        % Caption Part C
        \node[color={black!60},inner sep=0,rotate=90,align=center] at ($(l1) + (-0.8,0)$) {\small Evaluation /\\Inference};
        % Arrow connection llm train to llm inference.
        \draw[->,color={black!30}] (llm) |- ($(llm)!0.5!(llm1)$) -| (llm1);
        % Legend
        \node[doc,fill={black!50}] (docLegend) at ($(root) + (10.8, -1.5)$) {};
        \node[anchor=west] (docDesc) at ($(docLegend) + (0.2, 0)$) {\scriptsize Document};
        \node[concept] (conceptLegend) at ($(docLegend) + (0, -0.6)$) {};
        \node[anchor=west] (conceptDesc) at ($(conceptLegend) + (0.2, 0)$) {\scriptsize Concept};
        \node (arrow1) at ($(docLegend.west) + (-0.05, -1.08)$) {};
        \node (arrow2) at ($(arrow1) + (0.65, 0)$) {};
        \draw[->] (arrow1) -- (arrow2);
        \node[anchor=west] (arrowDesc) at ($(conceptDesc.west) + (0, -0.47)$) {\scriptsize Is-a relation};
    \end{tikzpicture}

    \caption{\name: Using annotations of documents with their relevant concepts, we train an LLM to model relevant subgraphs of the target ontology with a custom regulariser. During inference, the generated subgraphs for each document are summed and pruned to give the final output ontology. For evaluation, we measure the similarity between the generated ontology and the ground truth.}
    \label{fig:overview}
\end{figure}