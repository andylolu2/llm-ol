\chapter{Introduction}

\label{firstcontentpage} % start page count here

An ontology is a formal and structural way of representing domain-specific concepts and their relations~\cite{gruber1995toward}.
They can be simplistic consisting of \emph{concepts} and only a small number of types of \emph{taxonomic relations} (e.g., \emph{is-a} relationships). Or they can be complex consisting of axioms and many types of relations. For example, a simple ontology for programming languages might be:
\begin{figure}[H]
    \centering
    \begin{tikzpicture}[>=latex]
        \node[rectangle, draw] (dynamically-typed) {Dynamically-typed language};
        \node[rectangle, draw, right=of dynamically-typed] (python) {Python};
        \draw[->] (dynamically-typed) -- (python);
    \end{tikzpicture}
\end{figure}
\vspace{-1em}
It has two concepts and one relation, representing the knowledge that Python is a dynam\-ically-typed language. A more complex ontology might contain axioms too, for example, ``all programming languages are either dynamically or statically typed''. In this project, I focus on ontologies with only concepts and taxonomic relations. Compared to typical deep learning models which represent knowledge implicitly in its weights, ontologies capture knowledge in a structured and explicit manner, making them reliable, easy to edit and human-interpretable. Such benefits of ontologies have led to their wide adoption in practice. For example, the Wikipedia categories have been used for entity ranking~\cite{vercoustre2008using} and information retrieval~\cite{sorg2012exploiting} or the Schema.org~\cite{Schema.org_2011} ontology which is part of the Semantic Web~\cite{antoniou2004semantic} initiative.

While ontologies are useful, building ontologies often requires substantial manual effort. Ontology learning (OL) is the study of automating the construction of high-quality ontologies at scale. For a simplistic ontology, this amounts to discovering the concepts and taxonomic relations, usually based on a source corpus. This project aims contribute to this field by researching domain-independent methods for OL that are scalable and produce better ontologies.

\section{Project goals}

Traditionally, OL is viewed as a composition of subtasks~\cite{asim2018survey}, such as concept discovery and relation extraction. In particular, prior works have demonstrated that state-of-the-art large language models (LLMs) can solve such subtasks effectively~\cite{babaei2023llms4ol}. While studying subtasks permits fine-grained analysis and evaluation, it does not directly indicate the subsequent impact on the final ontology. Moreover, there is potential room for improvement by combining several subtasks into one. In this project, I instead develop and evaluate methods that construct ontologies in an end-to-end fashion to answer the following research questions:
\begin{enumerate}
    \item How can we leverage LLMs' knowledge base to build ontologies from scratch?
    \item Does our method scale efficiently to practical problem sizes?
    \item How well does our method generalise to new domains?
\end{enumerate}

\section{Achievements}

% \begin{figure}[t]
    \centering
    \begin{tikzpicture}
        [
            >=latex,
            doc/.style={
                    rectangle,
                    rounded corners=2pt,
                    % fill={black!80},
                    minimum width=10pt,
                    minimum height=14pt,
                    outer sep=2pt,
                },
            llm/.style={
                    rectangle,
                    rounded corners=2pt,
                    fill={black!20},
                    minimum width=40pt,
                    minimum height=24pt,
                    outer sep=3pt,
                },
            concept/.style={
                    circle,
                    fill={black!30},
                    inner sep=1.25pt,
                    outer sep=2.5pt,
                },
            every node/.append style={font=\sffamily},
            pics/.cd,
            % Marque croix en diagonale
            Cross/.style args={#1 and #2}{%
                    code = {%
                            \draw[#2,rotate=45,scale=1.4,very thick]
                            (0,#1 pt) -- (0,-#1 pt) ;
                            \draw[#2,rotate=-45,scale=1.4,very thick]
                            (0,#1 pt) -- (0,-#1 pt) ;
                        }
                },
            Cross/.default={2.5 and gray!25!black},
        ]
        \newcommand{\nodeDist}{0.7}
        \newcommand{\nodeDistShort}{0.55}
        \newcommand{\angleA}{30}
        \newcommand{\angleB}{66}
        \definecolor{c0}{RGB}{150, 150, 150}
        \definecolor{c1}{RGB}{255, 93, 82}
        \definecolor{c2}{RGB}{94, 118, 255}
        \definecolor{c3}{RGB}{124, 188, 220}
        \definecolor{c4}{RGB}{215, 140, 255}
        \definecolor{c5}{RGB}{255, 195, 84}
        \definecolor{c6}{RGB}{64, 194, 62}

        % PART A: Ontology
        \node[concept,color=c0] (root) {};
        \node[concept,color=c4] (l1) at ($(root) + ({-90+\angleA}:{\nodeDist})$) {};
        \node[concept,color=c6] (l2) at ($(root) + ({90-\angleA}:{\nodeDist})$) {};
        \node[concept,color=c1] (l11) at ($(l1) + ({90-\angleB}:{\nodeDist})$) {};
        \node[concept,color=c2] (l12) at ($(l1) + ({-90+\angleB}:{\nodeDist})$) {};
        \node[concept,color=c3] (l21) at ($(l2) + ({90-\angleB}:{\nodeDist})$) {};
        \node[concept,color=c5] (l22) at ($(l2) + ({-90+\angleB}:{\nodeDist})$) {};
        % \node[concept] (l1) at ($(root) + ({180+\angleA}:{\nodeDist})$) {};
        % \node[concept] (l2) at ($(root) + ({-\angleA}:{\nodeDist})$) {};
        % \node[concept] (l11) at ($(l1) + ({180+\angleB}:{\nodeDist})$) {};
        % \node[concept] (l12) at ($(l1) + ({-\angleB}:{\nodeDist})$) {};
        % \node[concept] (l21) at ($(l2) + ({180+\angleB}:{\nodeDist})$) {};
        % \node[concept] (l22) at ($(l2) + ({-\angleB}:{\nodeDist})$) {};
        \draw[->] (root) -- (l1);
        \draw[->] (root) -- (l2);
        \draw[->] (l1) -- (l11);
        \draw[->] (l1) -- (l12);
        \draw[->] (l2) -- (l21);
        \draw[->] (l2) -- (l22);

        \node[anchor=west,inner sep=0,align=left] (caption) at ($(root.west) + (0.4, 1.9)$) {\small Dataset};

        \newcommand{\docSpace}{0.66}
        \node[doc,fill=c1] (doc1) at ($(l11) + (\docSpace, 0)$) {};
        \node[doc,fill=c2] (doc2) at ($(l12) + (\docSpace, 0)$) {};
        \node[doc,fill=c3] (doc3) at ($(l21) + (\docSpace, 0.26)$) {};
        \node[doc,fill=c4] (doc4) at ($(l11) + (\docSpace, -1.5)$) {};
        \node[doc,fill=c5] (doc7) at ($(l22) + (\docSpace, 0)$) {};
        \node[doc] (doc5) at ($(doc2)!0.5!(doc4)$) {...};
        \node[doc] (doc6) at ($(doc3)!0.5!(doc7)$) {...};
        \draw[dashdotted] (l11) -- (doc1);
        \draw[dashdotted] (l12) -- (doc2);
        \draw[dashdotted] (l21) -- (doc3);
        \draw[dashdotted] (l22) -- (doc7);
        \draw[dashdotted,bend right] (l1) to (doc4);

        % line to separate the ontology into train and test
        % \draw[dashed] ($(root) + (0, 0.4)$) -- ($(root) + (0, -2.6)$);
        % \node at ($(root) + (1.1, -0.1)$) {Train};
        % \node at ($(root) + (-1.1, -0.1)$) {Test};

        % \draw[dashed] ($(root) + (-0.4, 0)$) -- ($(root) + (2.0, 0)$);
        % \node[anchor=west,inner sep=0] at ($(root.west) + (0, 1.1)$) {\small Train};
        % \node[anchor=west,inner sep=0] at ($(root.west) + (0, -1.5)$) {\small Test};

        % PART B: Training
        \newcommand{\trainSpace}{0.7}
        \node[doc,fill=c3] (input) at ($(doc3) + (1.0, -0.3)$) {};

        \node[llm,anchor=west] (llm) at ($(input.east) + (\trainSpace, 0)$) {\small LLM};
        \node[anchor=west,align=center] (loss) at ($(llm.east) + (\trainSpace, 0)$) {\small Mask-regularised loss};
        \begin{scope}[local bounding box=sg,shift={($(loss.east) + (0.7, 0.4)$)}]
            \node[concept,color=c0] (sgRoot) {};
            \node[concept,color=c6] (sg2) at ($(sgRoot) + ({-\angleA}:{\nodeDistShort})$) {};
            \node[concept,color=c3] (sg21) at ($(sg2) + ({-\angleB}:{\nodeDistShort})$) {};
            \draw[->] (sgRoot) -- (sg2);
            \draw[->] (sg2) -- (sg21);
        \end{scope}
        \draw[->] (input) -- (llm);
        \draw[->] (llm) -- (loss);
        \draw[->] (sg) -- (loss);
        \draw[->,bend right] (loss) to node [midway,above,align=center] {\scriptsize Backpropagate} (llm);
        \node at ($(input) + (0, 0.6)$) {\scriptsize Input};
        \node at ($(sg) + (0, 0.8)$) {\scriptsize Target};

        % PART C: Inference
        \node[llm] (llm1) at ($(llm) + (0, -2.3)$) {\small LLM};
        \node[doc,fill=c1,anchor=east] (input1) at ($(llm1.west) + (-\trainSpace, 0.8)$) {};

        \begin{scope}[local bounding box=sg2,shift={($(llm1.east) + (\trainSpace + 0.8, 1.1)$)}]
            \node[concept,color=c0] (sgRoot2) {};
            \node[concept,color=c4] (sg21) at ($(sgRoot2) + ({180+\angleA}:{\nodeDistShort})$) {};
            \node[concept,color=c1] (sg212) at ($(sg21) + ({-\angleB}:{\nodeDistShort})$) {};
            \draw[->] (sgRoot2) -- (sg21);
            \draw[->] (sg21) -- (sg212);
        \end{scope}

        \begin{scope}[local bounding box=sg1,shift={($(llm1.east) + (\trainSpace + 0.8, -0.45)$)}]
            \node[concept,color=c0] (sgRoot1) {};
            \node[concept,color=c4] (sg11) at ($(sgRoot1) + ({180+\angleA}:{\nodeDistShort})$) {};
            \node[concept,color=c6] (sg12) at ($(sgRoot1) + ({-\angleA}:{\nodeDistShort})$) {};
            \node[concept,color=c2] (sg111) at ($(sg11) + ({180+\angleB}:{\nodeDistShort})$) {};
            \node[doc,fill=c2,anchor=east] (input2) at ($(llm1.west) + (-\trainSpace, -0.7)$) {};
            \draw[->] (sgRoot1) -- (sg12);
            \draw[->] (sgRoot1) -- (sg11);
            \draw[->] (sg11) -- (sg111);
        \end{scope}

        \node (dots) at ($(input1)!0.5!(input2)$) {...};
        \node (dots2) at ($(dots) + (4.0, 0)$) {...};
        \newcommand{\vertSpace}{0.25}
        \draw (input1) -| ($(input1.east)!0.5!(llm1.west)$) |- ($(llm1.west) + (0, \vertSpace)$);
        \coordinate (out1) at ($(llm1.east) + (\trainSpace, 0.8)$);
        \draw[->] ($(llm1.east) + (0, \vertSpace)$) -| ($(llm1.east)!0.5!(out1)$) |- (out1);
        \draw (input2) -| ($(input2.east)!0.5!(llm1.west)$) |- ($(llm1.west) + (0, -\vertSpace)$);
        \coordinate (out2) at ($(llm1.east) + (\trainSpace, -0.8)$);
        \draw[->] ($(llm1.east) + (0, -\vertSpace)$) -| ($(llm1.east)!0.5!(out2)$) |- (out2);


        % \node[doc,fill=c1] (input1) at ($(input) + (0, -2.0)$) {};
        % \node[llm] (llm1) at ($(input1) + (\trainSpace, 0)$) {LLM};
        % \node[concept] (sgRoot1) at ($(llm1) + (\trainSpace + 0.8, 0.6)$) {};
        % \node[concept] (sg11) at ($(sgRoot1) + ({180+\angleA}:{\nodeDist})$) {};
        % \node[concept] (sg12) at ($(sgRoot1) + ({-\angleA}:{\nodeDist})$) {};
        % \node[concept] (sg111) at ($(sg11) + ({180+\angleB}:{\nodeDist})$) {};
        % \draw[->] (input1) -- (llm1);
        % \draw[->] (llm1) -- ($(llm1) + (\trainSpace-0.25, 0)$);

        % \node[doc,fill=c2] (input2) at ($(input1) + (0, -1.7)$) {};
        % \node[llm] (llm2) at ($(input2) + (\trainSpace, 0)$) {LLM};
        % \node[concept] (sgRoot2) at ($(llm2) + (\trainSpace + 0.8, 0.6)$) {};
        % \node[concept] (sg21) at ($(sgRoot2) + ({180+\angleA}:{\nodeDist})$) {};
        % \node[concept] (sg212) at ($(sg21) + ({-\angleB}:{\nodeDist})$) {};
        % \draw[->] (input2) -- (llm2);
        % \draw[->] (llm2) -- ($(llm2) + (\trainSpace-0.25, 0)$);

        \node[concept,color=c0] (testOutRoot) at ($(input1)!0.5!(input2) + (7.5, 0.4)$) {};
        \node[concept,color=c4] (testOut1) at ($(testOutRoot) + ({180+\angleA}:{\nodeDist})$) {};
        \node[concept,color=c1] (testOut11) at ($(testOut1) + ({-\angleB}:{\nodeDist})$) {};
        \node[concept,color=c2] (testOut12) at ($(testOut1) + ({180+\angleB}:{\nodeDist})$) {};
        \node[concept,color=c6] (testOut2) at ($(testOutRoot) + ({-\angleA}:{\nodeDist})$) {};
        \node at ($(testOutRoot) + (0, 0.45)$) {\scriptsize Output};
        \draw[->] (testOutRoot) -- (testOut1);
        \draw[->] (testOut1) -- (testOut11);
        \draw[->] (testOut1) -- (testOut12);
        \draw[->] (testOutRoot) -- pic[midway,-,rotate=-\angleA] {Cross={3.5 and red}} (testOut2);
        \draw[->] ($(input1)!0.5!(input2) + (4.6, 0)$) -- node [midway,above,align=center] {\scriptsize Sum and prune} ($(input1)!0.5!(input2) + (6.5, 0)$);

        % PART D: Evaluation
        \coordinate (start) at ($(l1) + (0, -0.8)$);
        \coordinate (inter) at ($(start) + (5.5, -1.5)$);
        \coordinate (end) at ($(testOutRoot) + (0, -1.2)$);
        \draw[<->,rounded corners] (start) |- (inter) -| (end);
        \node[below] at (inter) {\small Gold standard evaluation};

        % Draw a separation line between Part B and Part C
        \draw[color={black!30},dashed] ($(root) + (-0.8, 0)$) -- ($(root) + (13.2, 0)$);
        % Caption Part B
        \node[color={black!60},anchor=west,inner sep=0,rotate=90] at ($(root) + (-0.5, 0.2)$) {\small Training};
        % Caption Part C
        \node[color={black!60},anchor=west,inner sep=0,rotate=90,anchor=east] at ($(root) + (-0.5,-0.5)$) {\small Evaluation};
        % Arrow connection llm train to llm inference.
        \draw[->,color={black!30}] (llm) |- ($(llm)!0.5!(llm1)$) -| (llm1);
        % Legend
        \node[doc,fill={black!50}] (docLegend) at ($(root) + (11.6, -1.9)$) {};
        \node[anchor=west] (docDesc) at ($(docLegend) + (0.2, 0)$) {\scriptsize Document};
        \node[concept] (conceptLegend) at ($(docLegend) + (0, -0.6)$) {};
        \node[anchor=west] (conceptDesc) at ($(conceptLegend) + (0.2, 0)$) {\scriptsize Concept};
        \node (arrow1) at ($(docLegend.west) + (-0.05, -1.08)$) {};
        \node (arrow2) at ($(arrow1) + (0.65, 0)$) {};
        \draw[->] (arrow1) -- (arrow2);
        \node[anchor=west] (arrowDesc) at ($(conceptDesc.west) + (0, -0.47)$) {\scriptsize is-a relation};
    \end{tikzpicture}

    \caption{Overview of \name. Using annotations of documents with their relevant concepts, we train an LLM to model relevant subgraphs of the target ontology with a custom regulariser. During inference, the generated subgraphs for each document are summed and pruned to give the final output ontology. For evaluation, we measure the similarity between the generated ontology and the ground truth. \todo{Label `target' in training, `output/generated' in inference, rotate ontology 90 deg? add visual cues to the concepts. Align the documents on the same line}}
    \label{fig:overview}
\end{figure}

I introduce \name, an end-to-end method for using LLMs to construct ontologies at scale. Rather than focusing on individual relations between concepts, I finetune an LLM to model entire sub-components of the target ontology. The output ontology is generated by taking the sum of generated sub-components and applying simple post-processing. An overview of the pipeline is shown in \cref{fig:overview}. To train \name, I collect the categorisation metadata for a subset of Wikipedia articles. I attempt to adapt an LLM to model the relevant categorisation subgraph for a particular Wikipedia article, but discover that direct finetuning leads to poor generalisation due to overfitting to high-level, frequently occurring concepts. Instead, I propose a custom regulariser that reweights each concept based on its frequency of occurrence, which substantially improves generalisation.

I evaluate \name by measuring the similarity of the generated ontology with the ground truth. Current approaches for comparing ontologies rely on mapping classes of the two ontologies onto each other, most commonly by literal text matching. \todo{Add citation} This is unreliable when the two ontologies are not already sufficiently similar. Instead, I propose a suite of evaluation metrics suitable for comparing arbitrary labelled graphs. These metrics compare edges and subgraphs of the two ontologies using pretrained text embedders to test for semantic and structural similarity. The results reveal that an LLM can already outperform existing extraction-based methods out of the box, and the performance can be further improved by finetuning with our custom regulariser. I additionally demonstrate that \name can be adapted to build the arXiv ontology using only a small number of training examples, suggesting that our model can be applied to new domains in a data-efficient way.

\textbf{Contributions}
\begin{enumerate}
    \item I constructed two datasets based on Wikipedia and arXiv, which can serve as standard datasets for future work studying end-to-end OL.
    \item I created \name, a method that utilises LLMs to build ontologies from scratch. \name produces high-quality ontologies and serves as a strong baseline for end-to-end OL.
    \item I developed new evaluation metrics for end-to-end OL.
\end{enumerate}
