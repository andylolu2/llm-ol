\chapter{Background}

A more extensive coverage of what's required to understand your work.
In general you should assume the reader has a good undergraduate
degree in computer science, but is not necessarily an expert in the
particular area you've been working on. Hence this chapter may need to
summarize some ``text book'' material.

This is not something you'd normally require in an academic paper, and
it may not be appropriate for your particular circumstances. Indeed,
in some cases it's possible to cover all of the ``background''
material either in the introduction or at appropriate places in the
rest of the dissertation.

% \chapter{Related work}

This chapter covers relevant (and typically, recent) research
which you build upon (or improve upon). There are two complementary
goals for this chapter:
\begin{enumerate}
    \item to show that you know and understand the state of the art; and
    \item to put your work in context
\end{enumerate}

Ideally you can tackle both together by providing a critique of
related work, and describing what is insufficient (and how you do
better!)

The related work chapter should usually come either near the front or
near the back of the dissertation. The advantage of the former is that
you get to build the argument for why your work is important before
presenting your solution(s) in later chapters; the advantage of the
latter is that don't have to forward reference to your solution too
much. The correct choice will depend on what you're writing up, and
your own personal preference.