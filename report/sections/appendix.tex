\appendix

\chapter{Experiment details}  \label{appendix:exp-details}

\section{\name training}  \label{appendix:training-details}

For the Wikipedia experiment, we use Mistral 7B v0.2 (not instruction-tuned) \cite{jiang2023mistral} as the base model. We attach LoRA \cite{hu2021lora} adaptors to all attention and feed-forward layers with parameters $r=32$ and $\alpha=16$. The model is trained for 2 epochs ($\approx$ 17K steps) with batch size 16, context length 2048, and is optimised with Adam using a constant learning rate of 1e-5 with warm-up from zero for the first 100 steps. \textbf{Finetune} uses the same configuration. Training on two A100 GPUs takes $\approx 6$ hours.

For the arXiv experiment, we further finetune the model trained on Wikipedia with masked loss objective on 2048 document-subgraph pairs from the arXiv training set. We merge the LoRA adaptors from the Wikipedia experiment and initialise new ones with $r=8$ and $\alpha=8$. The model is trained with batch size 16 and Adam with constant learning rate 3e-6 and warp-up from zero for the first 10 steps. Early stopping is used to terminate training when the loss stops improving on the evaluation set, which happened at step 288. \textbf{Finetune (transfer)} uses the same configuration. Eearly stopping happened at step 192.

\begin{figure}[h]
    \centering
    \begin{lstlisting}[frame=single]
<s>[INST]\
Title: {{ title }}
{{ abstract }}[/INST]\

{{ path | join(" -> ") }}
\
</s>
\end{lstlisting}
    \caption{Linearisation template for \name training.}
    \label{fig:linearisation-template}
\end{figure}

\section{Hearst}

The \textbf{Hearst} baseline follows the implementation by \citet{roller2018hearst}. We give a description of the implementation here. \todo{CoreNLP, blah blah blah...}

\section{REBEL}

\section{Prompting}
We sample the one/three-shot examples from the training set for each query. The output is parsed using regex and results that do not match the regex are discarded.

\section{Prompt templates}  \label{appendix:prompt-template}

\subsection*{\name finetuning template}

\begin{lstlisting}[frame=single]
<s>[INST]\
Title: {{ title }}
{{ abstract }}[/INST]\

{{ path | join(" -> ") }}
\
</s>
\end{lstlisting}

\subsection*{Zero, one, and three-shot prompt template}

\begin{lstlisting}[frame=single]
The following is an article's title and abstract. Your task is to assign this article to suitable category hierarchy. A category is typically represented by a word or a short phrase, representing broader topics/concepts that the article is about. A category hierarchy represented by a collection of paths from the generic root category "Main topic classifications" to a specific category suitable for the article. The topics titles should become more and more specific as you move from the root to the leaf. 



### EXAMPLE {{ loop.index }} ###
### ARTICLE ###
Title: {{ example['title'] }}
{{ example['abstract'] }}
### END ARTICLE ###

{{ path | join(" -> ") }}

### END EXAMPLE {{ loop.index }} ###


You must answer in the format of:
Main topic classifications -> Broad topic 1 -> Subtopic 1 -> ... -> Most specific topic 1
Main topic classifications -> Borad topic 2 -> Subtopic 2 -> ... -> Most specific topic 2
...


### ARTICLE ###
Title: {{ title }}
{{ abstract }}
### END ARTICLE ###

Provide a category hierarchy for the above article. \

Use the same format as the examples above.

Use the format described above.

\end{lstlisting}

\newpage

\chapter{Visualisations} \label{appendix:visualisation}

% \subsubsection{Ground truth}

\begin{figure}[h]
    \centering
    \includegraphics[width=0.9\linewidth]{media/arxiv_v2_test_gt_graph.pdf}
    \caption{Ground truth test split ontology for arXiv}
\end{figure}

% \subsubsection{\name}

\begin{figure}[h]
    \centering
    \includegraphics[width=0.9\linewidth]{media/arxiv_v2_Finetune masked (transfer)_graph.pdf}
    \caption{Ontology for arXiv generated by \name}
\end{figure}

% \subsubsection{Finetune}

\begin{figure}[h]
    \centering
    \includegraphics[width=\linewidth]{media/arxiv_v2_Finetune (transfer)_graph.pdf}
    \caption{Ontology for arXiv generated by Finetune}
\end{figure}

% \subsubsection{Memorisation}

\begin{figure}[h]
    \centering
    \includegraphics[width=0.9\linewidth]{media/arxiv_v2_Memorisation_graph.pdf}
    \caption{Ontology for arXiv generated by Memorisation}
\end{figure}

% \subsubsection{Hearst}

\begin{figure}[h]
    \centering
    \includegraphics[width=0.9\linewidth]{media/arxiv_v2_Hearst better_graph.pdf}
    \caption{Ontology for arXiv generated by Hearst}
\end{figure}

% \subsubsection{REBEL}

\begin{figure}[h]
    \centering
    \includegraphics[width=0.8\linewidth]{media/arxiv_v2_Rebel better_graph.pdf}
    \caption{Ontology for arXiv generated by REBEL}
\end{figure}

% \subsubsection{Prompting}

\begin{figure}[h]
    \centering
    \includegraphics[width=0.8\linewidth]{media/arxiv_v2_0 shot_graph.pdf}
    \caption{Ontology for arXiv generated by zero-shot}
\end{figure}

\begin{figure}[h]
    \centering
    \includegraphics[width=0.9\linewidth]{media/arxiv_v2_1 shot_graph.pdf}
    \caption{Ontology for arXiv generated by one-shot}
\end{figure}

\begin{figure}[h]
    \centering
    \includegraphics[width=0.9\linewidth]{media/arxiv_v2_3 shot_graph.pdf}
    \caption{Ontology for arXiv generated by three-shot}
\end{figure}

% \begin{figure}
%     \centering
%     \includegraphics[width=\linewidth]{media/arxiv_ground_truth.pdf}
%     \caption{Ground truth ontology for arXiv}
%     \label{fig:arxiv-ground-truth}
% \end{figure}